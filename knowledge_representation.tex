\section{Knowledge Representation}

Factory floor maintenance can be split into categories with regards to a variety of parameters. For this project, we have split this maintenance into 2 types: \acrfull{pm} and \acrfull{pdm}.

\subsection{Preventive Maintenance}

\acrshort{pm} is what is normally considered as scheduled maintenance. This type of maintenance is characteristically:

\begin{itemize}
	\item Planned at regular intervals
	\item Usually requires machine downtime
	\item Often consists of a checklist of tasks to complete
	\item May exist has a part of different maintenance regimens
	\item Occurs even if there are identifiable issues present
\end{itemize}

This type of maintenance has key benefits over other types of maintenance, such as being proactive and efficient as well as limiting unplanned downtime and increasing equipment lifespan.

\subsection{Predictive Maintenance}

On the other hand, \acrshort{pdm} takes advantage of advances in connectivity and data collection to monitor equipment in real time, enabling the identification of potential for issues before they occur. \acrshort{pdm} is characterized by:

\begin{itemize}
	\item Being proactive
	\item Being performed as the machines are running in production mode
	\item Identifies and addresses potential problems before failure
	\item Relies on measurement and data collection systems, as well as tools and classified personnel to analyze and act on the obtained data.
\end{itemize}

This type of maintenance has some advantages over \acrshort{pm}, such as being able to identify issues so they can be addressed, allowing for shorter downtime, improving inventory efficiency and making available different maintenance practices.

However, it also provides an array of challenges, such has complexity, less schedulability compared to \acrshort{pm}, more expensive equipment and/or infrastructure as well as more personnel or training, when compared to \acrshort{pm}.



