%!TEX root = main.tex

\section{Fontes de Conhecimento}

De forma a proceder à concretização dos nossos objetivos, foi necessário obter conhecimento pericial no tópico sobre investigação. A nossa pesquisa levou-nos a obter este conhecimento a partir de uma série de fontes. Obtemos este conhecimento maioritariamente a partir do perito, que apoiou o nosso projeto, do material de pesquisa providenciado pelo mesmo e pela pesquisa bibliográfica autónoma efetuada pelos membros da equipa.

\subsection{Perito}

Devido à relativa novidade do tópico deste projeto, foi verificada uma dificuldade generalizada em encontrar um perito com conhecimento na área e com disponibilidade para nos apoiar. Devido a este percalço, e com a ajuda do professor Carlos Ramos, foi possível encontrar um perito, o professor Jorge Meira.

Selecionamos este perito devido à sua participação de em projetos associados a sistemas \acrshort{iot} e desenvolvimento de artigos científicos na área de intrusão a sistemas computacionais \parencite{meira2017}, para além da recomendação do professor Carlos Ramos.

\subsection{Artigos}

Antes de reunir com o perito, foi efetuada uma pesquisa bibliográfica aprofundada sobre o tópico de comunicação \acrshort{iot}, deteção de intrusões e utilização de sistemas \acrshort{iot} no contexto da Indústria 4.0. Esta pesquisa permitiu-nos interagir corretamente com o perito, enquadrando o conhecimento disponibilizado pelo mesmo.

\subsection{Conjunto de Dados}
\label{sec:dataset}

De forma a obter dados reais sobre o tópico em pesquisa, foi necessário obter informação relativa a casos de intrusão em sistemas \acrshort{iot}. Para este efeito foi utilizado um \textit{dataset} indicado pelo perito, Aposemat IOT-23 \parencite{sebastian_garcia_2020_4743746}. Este conjunto de dados contém tráfego de rede (maligno e benigno) no contexto de dispositivos \acrshort{iot}, tendo sido produzido pelo laboratório Avast-AIC \parencite{sebastian_garcia_2020_4743746}.