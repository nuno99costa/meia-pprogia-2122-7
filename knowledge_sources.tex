\section{Fontes de Conhecimento}

De forma a proceder à concretização dos nossos objetivos, foi necessário obter conhecimento pericial no tópico sobre investigação. A nossa pesquisa levou-nos a obter este conhecimento a partir de uma série de fontes. Obtemos este conhecimento de forma maioritária a partir do perito, que apoiou o nosso projeto, do material de pesquisa providenciado pelo mesmo e pela pesquisa bibliográfica autónoma efetuada pelos membros da equipa.

\subsection{Perito}

\textcolor{red}{\textbf{FALTA FALAR SOBRE O PERITO}}

\subsection{Artigos}

Antes de reunir com o perito, foi efetuada uma pesquisa bibliográfica aprofundada sobre o tópico de comunicação IoT, deteção de intrusões e utilização de sistemas IoT no contexto da Indústria 4.0. Esta pesquisa permitiu-nos interagir corretamente com o perito, enquadrando o conhecimento disponibilizado pelo mesmo.

\subsection{Conjunto de Dados}

De forma obter dados reais sobre o tópico em pesquisa, foi necessário obter informação relativa a casos de intrusão em sistemas IoT. Para este efeito foi utilizado um \textit{dataset} indicado pelo perito, Aposemat IoT-23 \textbf{citehere}. Este conjunto de dados contém tráfego de rede (maligno e benigno) no contexto de dispositivos IoT, tendo sido produzido pelo laboratório Avast-AIC \textbf{citehere}.